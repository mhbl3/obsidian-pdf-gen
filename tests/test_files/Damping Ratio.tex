\documentclass[10pt]{report}
\usepackage{minted}
\usepackage{enumitem}
\setlistdepth{9}

\setcounter{secnumdepth}{3}

\usepackage{titlesec}
\titleformat{\chapter}[display]{\normalfont\huge\bfseries}{}{0pt}{\Huge}
\titleformat{name=\chapter,numberless}[display]
  {\normalfont\huge\bfseries}{}{0pt}{\Huge}

\setlist[itemize,1]{label=\textbullet}
\setlist[itemize,2]{label=\textbullet}
\setlist[itemize,3]{label=\textbullet}
\setlist[itemize,4]{label=\textbullet}
\setlist[itemize,5]{label=\textbullet}
\setlist[itemize,6]{label=\textbullet}
\setlist[itemize,7]{label=\textbullet}
\setlist[itemize,8]{label=\textbullet}
\setlist[itemize,9]{label=\textbullet}

\renewlist{itemize}{itemize}{9}

\usepackage{graphicx}

\usepackage{csquotes}

\usepackage[hidelinks]{hyperref}

\usepackage[T1]{fontenc}

\usepackage[most]{tcolorbox}
\definecolor{block-gray}{gray}{0.95}
\newtcolorbox{advtcolorbox}{
    %colback=block-gray,
    boxrule=0pt,
    boxsep=0pt,
    breakable,
    enhanced jigsaw,
    borderline west={4pt}{0pt}{gray},
}

% Remove paragraph indentation
\setlength{\parindent}{0pt}

\newtcbox{\pill}[1][blue]{on line,
arc=7pt,colback=#1!10!white,colframe=#1!50!black,
before upper={\rule[-3pt]{0pt}{10pt}},boxrule=1pt,
boxsep=0pt,left=6pt,right=6pt,top=2pt,bottom=2pt}

\usepackage{amsmath}
\usepackage[dvipsnames]{xcolor} % to access the named colour LightGray
\usepackage{soul}
% \usepackage{sectsty}
\definecolor{LightGray}{rgb}{0.9, 0.9, 0.9}
\definecolor{inlinecodecolor}{rgb}{0, 0.3, 0.6}
\usepackage{lmodern}

\makeatletter
\renewcommand\subparagraph{%
\@startsection{subparagraph}{5}{0pt}%
{3.25ex \@plus 1ex \@minus .2ex}{-1em}%
{\normalfont\normalsize}}
\makeatother

\makeatletter
\renewcommand\paragraph{%
\@startsection{paragraph}{4}{0pt}%
{3.25ex \@plus -1ex \@minus .2ex}{-1em}%
{\normalfont\normalsize\bfseries}}
\makeatother
\sethlcolor{yellow}
\newrobustcmd*\inlinecode[2][]{\textcolor{inlinecodecolor}{\mintinline[#1]{python}{#2}}}
\newcommand{\myparagraph}[1]{\paragraph{\textcolor{black}{#1}}\mbox{}\\}
\newcommand{\mysubparagraph}[1]{\subparagraph{\textcolor{black}{#1}}\mbox{}\\}
\begin{document}


\chapter{Damping Ratio}\label{ch:Damping Ratio}

\pill{\#Damping-Ratio} 

\section{Definition: }

$$\xi = a/\omega_{n}$$

\begin{itemize}
\item $\xi$ is the damping ratio
\item $a$ is the rate of exponential decay

\begin{itemize}
\item For second order systems: $e^{-at}$ 

\end{itemize}
\item $\omega_{n}$ is the natural frequency of oscillation 

\item Measures the relative rate at which oscillation magnitudes are \textbf{reduced}, normalized by the natural frequency of the system

\item $\zeta < 1$ ->  underdamped

\begin{itemize}
\item Oscillatory response decays exponentially 

\item Lots of periods to eliminate the motion following a disturbance

\end{itemize}
\item $\zeta=0$ 

\begin{itemize}
\item Indefinitely oscillating response

\end{itemize}
\item $\lim_{\zeta \rightarrow 1} \zeta$

\begin{itemize}
\item Oscillations dissipate in relatively few periods of oscillatory motion

\end{itemize}
\item $\zeta=1$ 

\begin{itemize}
\item Critically damped system, with no motion

\end{itemize}
\end{itemize}


\subsection{Percent Overshoot Method}


$$\zeta = \frac{-\ln(\%OS/100)}{\sqrt{\pi^2 + \ln^2(\%OS/100)}}$$

\begin{itemize}
\item \hl{Response with too few peaks or too small to accurately extract from the time response}
\item $\%OS$: Percent overshoot to a step input

\begin{itemize}
\item Applying it to a system with additional poles and/or zeros is a good-way of determining “effective” damping ratio

\item \textbf{\hl{Large amount of overshoot but small residual oscillations}}

\begin{itemize}
\item Could be caused by presence of uncanceled zeros in the mid-frequency range (e.g. 1-10 rad/sec)

\item Perceived as lightly damped mode by pilot, especially in turbulence 

\end{itemize}
\end{itemize}
\end{itemize}

\begin{advtcolorbox}
Doesn’t exist for impulse resulting in a steady state value of zero 

\end{advtcolorbox}

\subsection{Logarithmic Decrement}

$$\zeta = \frac{\delta}{\sqrt{4\pi^2 + \delta^2}}$$

\begin{itemize}
\item Where $\delta$ (the natural logarithm of the ratio of successive peaks with period $T$ overreacted in measurement $y(t)$ :
\begin{itemize}
\item $\delta = \ln\left[\frac{y(t)}{y(t+T)}\right]$

\end{itemize}
\item Significant residual oscillations exists

\item Used peak times and magnitude to determine natural frequency and decay frequency of the response, from which $\zeta$ is calculated

\item Usually calculated from an impulse inputs and rate output (e.g., pitch rate)

\begin{itemize}
\item But step input should work

\end{itemize}
\item At least 3 oscillations required

\end{itemize}

\begin{advtcolorbox}
From Wikipedia: The method of logarithmic decrement \textbf{becomes less and less precise as the damping ratio increases past about 0.5}; it does not apply at all for a damping ratio greater than 1.0 because the system is \href{https://en.m.wikipedia.org/wiki/Overdamped}{overdamped}

\end{advtcolorbox}


\end{document}
